\documentclass{article}
\usepackage{graphicx} % Required for inserting images
\usepackage[top=0.9in, bottom=1in, left=1.5in, right=1.5in]{geometry}
\usepackage[utf8]{inputenc}
\usepackage[icelandic]{babel}
\usepackage[T1]{fontenc}
\usepackage[sc]{mathpazo}
\usepackage[parfill]{parskip}
\renewcommand{\baselinestretch}{1.2}
\usepackage{booktabs,tabularx}
\usepackage{multirow}
\usepackage{enumerate}
\usepackage{adjustbox}
\usepackage{multicol}
\usepackage{xcolor}
\usepackage{algpseudocode}
\usepackage{tikz}
\usepackage{nicefrac}
\usepackage{changepage}
\usetikzlibrary{arrows, positioning, calc, graphs}
\usepackage{amsmath, amsfonts, amssymb, amsthm}
\usepackage{graphicx}
\usepackage{tikz}
\usepackage{minted}
\usemintedstyle{manni}
\title{Einstaklingsverkefni}
\author{Ragnar Björn Ingvarsson, rbi3}
\tikzset{->, >=stealth', shorten >=1pt, node distance=2cm,thick, main node/.style={circle,draw,minimum size=3em}}

\begin{document}
\renewcommand\thepage{}

	\maketitle

	\newpage
	\setcounter{page}{1}
	\renewcommand\thepage{\arabic{page}}

	\section{Inngangur}
	Verkefnið er byggt á því að ég hef enga hugmynd um hvað ég vil gera, svo 
	ég ákvað að leyfa mér að hafa það svolítið opið um stund með því að ákveða 
	að nota GSAP til að gera eitthvað áhugavert.

	\section{Útfærsla}
	Vegna þess að verkefnið mitt er aðallega framendi, þá ætla ég semsagt að 
	uppfylla skilyrði sem tengjast framenda. Ég ætla að, í fyrsta lagi útfæra 
	framenda, einnig að útfæra RESTful virkni því ég hef dálítinn áhuga á 
	hvernig það virkar, og svo líka að hýsa verkefnið á einhverri síðu svo 
	hægt sé að skoða hana annarsstaðar en frá minni eigin tölvu. Mögulegt er 
	að ég útfæri eitthvað fleira en það mun koma í ljós seinna.

	\section{Tækni}

	Ég ætla aðallega að nota GSAP fyrir kvikun og mun líklega einnig nota 
	React fyrir framenda en langar einnig að prófa önnur framendaframework, 
	svo allt gæti gerst. Ég mun nota Node.js umhverfið þar sem ég er loksins 
	búinn að læra á það smá og ég nenni hreinlega ekki að læra á nýtt. Loks 
	mun ég svo örugglega hýsa verkefnið á Render þar sem það hefur reynst 
	mjög þægilegt í notkun hingað til.

	\section{Hvað gekk vel?}
	\section{Hvað gekk illa?}
	\section{Hvað var áhugavert?}
	
\end{document}
